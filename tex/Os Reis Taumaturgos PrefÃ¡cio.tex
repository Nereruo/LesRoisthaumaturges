\documentclass[a5paper]{book}
\usepackage[top=1.5cm, bottom=1.5cm, left=2.25cm, right=1.75cm]{geometry}
\usepackage[french, brazilian]{babel}
\usepackage[utf8]{inputenc}
\usepackage[T1]{fontenc}
\begin{document}
\frontmatter
\title{Os Reis Taumaturgos: ~\\O caráter sobrenatural do poder régio ~\\França e Inglaterra ~\\ ~\\Marc Bloch}
\author{~\\Prefácio: ~\\ Jacques Le Goff ~\\ ~\\ Tradução: ~\\ Júlia Mainardi ~\\ ~\\Digitalização:~\\Equipe Athelas}
\date{}
\maketitle
% Introductory chapters
%prefácio de Jacques Le Goff
\chapter{Prefácio}
Durante os aproximadamente trinta anos que se seguiram à morte heroica de Marc Bloch – torturado pela Gestapo, depois fuzilado aos 57 anos a 16 de julho de 1944, em Saint-Didier-de-Formants (Ain), perto de Lyon, por causa de sua participação na Resistência -, sua reputação como historiador teve tríplice fundamento, Primeiro, o papel de co-fundador e co-diretor, com Lucien Febvre, da revista \emph{Annales}\footnote{A revista, fundada em 1929 sob o título de \emph{\foreignlanguage{french}{Annales d'Histoire et Sociale}}, com a guerra passou a chamar-se \emph{\foreignlanguage{french}{Annales d'Histoire Sociale}} ~(1939-41 e 1945)e depois \emph{Mélanges d'Histoire Sociale} ~(1942-4), obedecendo às leis de Vichy, as quais principalmente exigiam que o nome judeu Marc Bloch desaparecesse da capa da Revista. De início, em maio de 1941, Marc Bloch manifestara numa carta a Lucien Febvre sua hostilidade a que a revista continuasse depois do estabelecimento do governo de Vichy; não obstante, colaborou nela sob o pseudônimo Marc Fougères e em outubro de 1942, em outra carta a Lucien Febvre, reviu sua desaprovação e reconheceu o acerto da decisão de Febvre. Sobre se tencionava retomar seu lugar após a guerra, os testemunhos são contraditórios. Em 1946, a revista passou a chamar-se \emph{Annales: Economies - Societés - Civilizations}, nome que tem até hoje.},a qual renovou os métodos históricos. Em seguida, dois grandes livros: \foreignlanguage{french}{Les caractères originaux de l’histoire rural française} [O caráter primordial da história rural francesa] (1931), apreciado principalmente pelos especialistas, que nele viram, com razão, o coroamento da história geográfica à francesa e o ponto de partida de uma nova visão da história rural na Idade Média e na época moderna; La société féodale [A sociedade Feudal] (1939-40), síntese eficiente e original que transfigurava a história das instituições por meio de uma concepção global da sociedade, integrando a história econômica, a história social e a história das mentalidades, e atingia um público mais amplo. A isso acrescentava-se um tratado (póstumo) sobre o método histórico, Apologie por l’histoire ou Métier d’historient [Apologia da história, ou Ofício de historiador] (publicado graças aos cuidados de Lucien Febvre em 1949), ensaio inacabado em que algumas percepções profundas e originais decompunham-se de tempos em tempos numa confusão que o autor decerto teria corrigido antes da publicação.

De alguns anos para cá, Marc Bloch é, para um número crescente de pesquisadores em ciências humanas e sociais, antes de tudo o autor de um livro pioneiro, seu primeiro verdadeiro livro, \emph{Les rois thaumaturges. Études sur le caractère attribué à la puissance royale, particulièrement en France et en Angleterre}~(1924), que faz desse grande historiador o fundador da antrolopologia histórica.\footnote{Foi isso que Georges Duby reconheceu em seu prefácio à 7ª edicção de \emph{Apologie pour l'histoire ou Métier d'historien}~(1974): "Quando aos 56 anos, nas últimas linhas que escreveu, o Bloch da Resistência afirma mais uma vez que as condições sociais são, 'em sua natureza produnda, mentais' ~(p. 158), não está ele nos convocando a retomar seu primeiro, seu verdadeiro grande livro, a reler \emph{Os reis taumaturgos} e a prosseguir essa história das mendalidades que ele abandonara, mar da qual o jovem Bloch, já precisamente cinquanta anos, foi talvez o inventor?" (p. 15).} 

~\\ \large \emph{GÊNESE DE "OS REIS TAUMATURGOS} ~\\

No estado atual de nossos conhecimentos sobre Marc Bloch, e esperando que o que se conservou de suas cartas e das de seus correspondentes talvez nos traga precisões, se não revelações, pode-se dizer que a gestação de \emph{Os reis taumaturgos} estendeu-se por uma dúzia de anos e beneficiou-se de três experiências principais, duas de ordem intelectual e no intervalo entre estas uma de ordem existencial.\footnote{Em primeiro lugar agradeço a Etienne Bloch, filho de Marc Bloch, ter colocado a meu dispor as informações e os documentos a respeito de \emph{Os reis taumaturgos} e ter-me autorizado a trabalhar sobre o fundo de papéis de Marc Bloch depositados nos Archives Nationales, os quais pude, graças à amabilidade da sra. Suzanne d'Huart, conservadora-chefe, consultar as melhores condições. Esse fundo traz o código AB XIX 3796-3852 (o código AB XIX deigna a documentação dos grandes eruditos depositadas nos Archives Nationales). A maior parte das citações deste Prefácio que não têm referências provém desse fundo. Também agradeço a meu amigo André Burguière diversas indicações preciosas.}

A primeira tem por teatro a Fondation Thiers, em Paris, onde Marc Bloch ~(que em 1908 saíra da École Normale Supérieure como professor agrégé de história) foi pensionista de 1909 a 1912. Depois vem a experiência da guerra de 1914-18, que ele terminou capitão, após ter sido citado quatro vezes por bravura e ter recebido a Croix de Guerre.

Enfim, deve-se considerar a atmosfera da faculdade de letras da Universidade de Estrasburgo, em que foi nomeado \emph{chargé de cours} em dezembro de 1919 e \emph{professeur} em 1921.

A atividade científica de Marc Bloch começa em 1911-2. Ele publica seus primeiros artigos. Até a guerra, esses estudos testemunham três centros de interesse, claramente ligados entre si. De início, a história institucional do feudalismo medieval, sobretudo o lugar da realeza e o da sevidão no sistema feudal, primeiros passos de um estudo que depois da guerra será paralisado (em virtude das disposições tomadas em favor dos universitários ex-combatentes) num embrião de tese? "Rois et serfs - un chapitre d'histoire capétienne" [Reis e servos - um capítulo de história capetíngia]. Em seguida (no quadro de geografia histórica que teve, a partir de Vidal de la Blanche e dos sucessores deste, influência tão grande sobre a nova escola histórica francesa do período entre as duas guerras), uma região: a Île-de-France. Enfim, uma primeira dissertação sobre o método: a pouquíssimo conhecida preleção pronunciada na distribuição dos prêmios do liceu de Amiens em 1914, às vesperas da Grande Guerra: "Critique historique et critique du témoignange" [Crítica histórica e crítica do testemunho].

\begin{sloppypar}Entre esses primeiros ensaios, um, que apareceu em 1912, merece atenção especial: \foreignlanguage{french}{"Les formes de la rupture de l'hommage dasns l'ancien droit féodal"}\footnote{Publicado na \emph{Nouvelle Revue Historique du Droit Français et Etranger}, t, XXXVI, mars-avril 1912, pp. 141-77, e reeditado em Marc Bloch, \emph{Mélanges historiques}, Paris, 1963 (Bibliothèque Générale de l'École Pratique des Hautes Études, vi\textsuperscript e \textnormal section, sepven), t. I, pp. 189-209.)} [Formas da ruptura da homenagem no antigo direito feudal], Marc Bloch descreve ali um "rito" feudal: o "arremesso da palha" e, às vezes, a "ruptura" da palha (\emph{exfestucatio}), significando, realizando a ruptura da homenagem. Interesse precoce, portanto, pelo ritual nas instituições do passado; e, ante a indiferença da quase totalidade dos historiadores em geral e dos historiadores do direito medieval francês em particular (duas notas de Gaston Paris e uma alusão de Jacques Flach), Marc Bloch volta-se para os historiadores alemães do direito medieval, então abertos à etnografia e ao comparativismo; um artigo de Ernst von Moeller e, sobretudo, "o grande trabalho do sr. Karl von Amira", \emph{Der Stab in der germanischen Rechtssynbolik}\footnote{Referências precisas sobre esses dois trabalhos encontram-se no citado artigo de Marc Bloch, \emph{Mélanges historiques,} I, p. 190, n. 2.} [O báculo no simbolismo legal germânico].\end{sloppypar}

~\\ \large \emph{O TRIO DA FONDATION THIERS} ~\\

Onde está então Marc Bloch? Depois de diversas passagens por universidades em 1908-9, em Berlim e em Leipzig, ele termina sua permanência na Fondation Thiers. Reencontrou ali dois antigos companheiros da École Normale, Lois Gernet, o helenista (formado em 1902), e Marcel Granet, o sinólogo (da turma de 1904, como Marc Bloch). Os três jovens eruditos organizaram entre si um pequeno grupo de pesquisas. Parece que a influência de Granet sobre seus dois amigos foi particularmente importante. A problemática e os métodos daquele que iria renovar a sinologia contribuíram a orientar Louis Gernet e Marc Bloch para percepções mais amplas que as da historiografia tradicional acerca da Grécia antiga e do Ocidente medieval. Antes que \emph{Os reis taumaturgos} apareça em 1924, Marcel Granet terá publicado \emph{Fêtes et chansons anciennes de la Chine} [Festas e canções antigas da China] (1919) e \emph{La religion des chinois} [A religião dos chineses] (1922) e  iniciado a reflexão e as pesquisas que o conduzirão às duas grandes sínteses: \emph{La civilization chinoise} [A civilização chinesa] (1929) e \emph{La penseé chinoise} [O pensamento chinês] (1934). Escreveu também \emph{La féodalite chinoise} [O feudalismo chinês], publicado em 1932 em Oslo, onde também viera a lume no ano anterior \emph{Les caractères originaux de l'histoire rurale française} de Marc Bloch, a quem Granet seguira à capital norueguesa como convidado estrangeiro do Instituto para o Estudo Comparado das Civilizações (apresentado por Marc Bloch nos \emph{Annales} em 1930, pp. 83-5). Desde suas primeiras fases, a obra de Granet contribuiu para confirmar o interesse de Marc Bloch pelos ritos e mitos, pelas cerimônias e lendas, pela psicologia coletiva comparada, pelos "sistemas de pensamento" e de crença das sociedades do passado\footnote{Como estudo (anterior a \emph{Os reis taumaturgos}) de um rito jurídico por Marcel Granet, ver "Le dépot de l'enfant sur le sol", publicado em 1922 em \emph{La Revue Archéologique}.}

Lois Gernet, cujo ensinamento ficou em seguida restrito, por muito tempo, à Universidade de Argel (verdade é que ele acolheu ali um jovem historiador chamado Fernand Braudel) e cuja obra foi escandalosamente marginalizada pelo helenismo universitário reinante, não está menos próximo de Marc Bloch por seu pensamento e por seu comportamento. A partir de 1917, Gernet publicou suas \emph{Recherches sur le développement de la pensée juridique et morale en Grèce} [Pesquisas sobre o desenvolvimento do pensamento jurídico e moral na Grécia]. Sua grande síntese, \emph{Le génie grec dans la religion} [O espírito grego na religião], escrita com André Boulanger para o período helenístico, é publicada em 1932 - mas sua notoriedade data apenas de sua reedição em 1970, quando a compilação póstuma de seus artigos, \emph{Anthopologie de la Grèce antique} {Antropologia da Grécia antiga} (1968, reeditada em 1982), permite enfim avaliar sua envergadura e compreender sua influência sobre a grande escola francesa contemporânea da antropologia histórica da Grécia antiga (Jean-Pierre Vernant; Pierr Vidal-Naquet; Marcel Detienne, vindo de Liège; Nicole Loraux; François Hartog; e outros). As discussões de Marc Bloch (e de Granet) com Gernet só fizeram aprofundar sua atenção para o etnolegalismo, o mito, o ritual, o comparatismo perspicaz e prudente.\footnote{Dessas informações sobre o grupo Bloc-Gernet-Granet na Fondation Thiers em 1909-12, devo o essencial a Ricardo Di Donato, professor da Scuola Normale Supeiore di Pisa, o qual prepara um grande trabalho sobre Louis Gernet e a quem agradeço calorosamente.}

~\\ \large \emph{A GRANDE GUERRA} ~\\

Depois, vem a segunda experiência: a guera de 1914-18. Para Marc Bloch, foi uma aventura extraordinária. As memórias que escreveu durante o primeiro ano do conflito mostram aliar com simplicidade um patriotismo ardente, um desejo de nada esconder a respeito das realidades sórdidas e cruéis da vida dos combatentes. Mas conserva sempre uma lucidez que lhe permite, mesmo durante a ação mais acirrada, contemplar com desprendimento a ação, lançar um olhar repleto de humanidade (ainda que sem condescendência) sobre os homens a seu redor e sobre si mesmo. Esforça-se constantemente para, como historiador, refletir sobre o que vê e sobre o que vive. No primeiro dia em que é lançado na batalha, 10 de setembro de 1914, observa que: "O senso de curiosidade, o qual raramente me abandona, não me deixara". À \emph{curiosidade}, primeiro estímulo da história, junta-se em seguida um trabalho de pesquisa da \emph{memória}. Sempre anota num caderno os acontecimentos do dia, até que, depois de 15 de novembro de 1914, um ferimento e a doença o impeçam de continuar esse registro de marcha. No início de 1915, quando uma doença grave faz que seja enviado para a retaguarda e obriga-o a um período de repouso para convalescença, ele apressa-se a escrever suas lembraças. Não quer ficar na dependência da memória: Esta opera no passado "uma triagem que frequentemente me parece pouco judiciosa". Ao final dessas recordações dos cinco primeiros meses de guerra, tira, na qualidade de historiador, as conclusões de sua experiência de combatente. Esboça os temas que retomará em 1940 en \emph{L'étrange défaite}\footnote{\emph{L'étrante défaite}, publicação póstima, Paris, 1946 (nova edição está sendo preparada pela Éditions Gallimard).} [A estranha derrata]. Mas para Marc Bloch o essencial é aquilo que concerne à psicologia, psicologia individual dos soldados e dos oficiais, psicologia coletiva dos grupos de guerreiros.\footnote{Ver Marc Bloc, "Souvenirs de guerre 1914-1915", \emph{Cahiers des Annales}, 26, Paris, 1969. Quando, como oficial, precisou garantir a defesa de homens levados a conselho de guerra, Marc Bloch pôde enriquecer sua experiência da psicologia do soldado. Dessas defesas se conservaram algumas notas. Ver o catálogo da Exposition Marc Bloch (preparado por André Burguière e Claude Chandonnay), École des Hautes Études en Sciences Sociales, mai 1979.}



\end{document}