% 26/07/2016 correeção das ligaturas "f i" e "f f" e "f l" com a inclusão do comando {\kern0pt} entre as duas letras sempre que aparecem juntas.
\documentclass[a5paper]{book}
\usepackage[top=1.5cm, bottom=1.5cm, left=2.25cm, right=1.75cm]{geometry}
\usepackage[brazilian]{babel}
\usepackage[utf8]{inputenc}
\usepackage[T1]{fontenc}
\begin{document}
\frontmatter
\title{Os Reis Taumaturgos: ~\\O caráter sobrenatural do poder régio ~\\França e Inglaterra ~\\ ~\\Marc Bloch}
\author{\\Prefácio: \\ Jacques Le Gof{\kern0pt}f \\ \\ Tradução: \\ Júlia Mainardi \\ \\Digitalização:\\Equipe Athelas}
\date{}
\maketitle
% Introductory chapters
%prefácio de Jacques Le Gof{\kern0pt}f
\chapter{Prefácio}
Durante os aproximadamente trinta anos que se seguiram à morte heroica de Marc Bloch – torturado pela Gestapo, depois fuzilado aos 57 anos a 16 de julho de 1944, em Saint-Didier-de-Formants (Ain), perto de Lyon, por causa de sua participação na Resistência -, sua reputação como historiador teve tríplice fundamento, Primeiro, o papel de co-fundador e co-diretor, com Lucien Febvre, da revista \textit{Annales}\footnote{A revista, fundada em 1929 sob o título de \textit{Annales d'Histoire et Sociale}, com a guerra passou a chamar-se \textit{Annales d'Histoire Sociale} ~(1939-41 e 1945)e depois \textit{Mélanges d'Histoire Sociale} ~(1942-4), obedecendo às leis de Vichy, as quais principalmente exigiam que o nome judeu Marc Bloch desaparecesse da capa da Revista. De início, em maio de 1941, Marc Bloch manifestara numa carta a Lucien Febvre sua hostilidade a que a revista continuasse depois do estabelecimento do governo de Vichy; não obstante, colaborou nela sob o pseudônimo Marc Fougères e em outubro de 1942, em outra carta a Lucien Febvre, reviu sua desaprovação e reconheceu o acerto da decisão de Febvre. Sobre se tencionava retomar seu lugar após a guerra, os testemunhos são contraditórios. Em 1946, a revista passou a chamar-se \textit{Annales: Economies - Societés - Civilizations}, nome que tem até hoje.},a qual renovou os métodos históricos. Em seguida, dois grandes livros: Les caractères originaux de l’histoire rural française [O caráter primordial da história rural francesa] (1931), apreciado principalmente pelos especialistas, que nele viram, com razão, o coroamento da história geográf{\kern0pt}ica à francesa e o ponto de partida de uma nova visão da história rural na Idade Média e na época moderna; La société féodale [A sociedade Feudal] (1939-40), síntese ef{\kern0pt}iciente e original que transf{\kern0pt}igurava a história das instituições por meio de uma concepção global da sociedade, integrando a história econômica, a história social e a história das mentalidades, e atingia um público mais amplo. A isso acrescentava-se um tratado (póstumo) sobre o método histórico, Apologie por l’histoire ou Métier d’historient [Apologia da história, ou Ofício de historiador] (publicado graças aos cuidados de Lucien Febvre em 1949), ensaio inacabado em que algumas percepções profundas e originais decompunham-se de tempos em tempos numa confusão que o autor decerto teria corrigido antes da publicação.

De alguns anos para cá, Marc Bloch é, para um número crescente de pesquisadores em ciências humanas e sociais, antes de tudo o autor de um livro pioneiro, seu primeiro verdadeiro livro, \textit{Les rois thaumaturges. Études sur le caractère attribué à la puissance royale, particulièrement en France et en Angleterre}~(1924), que faz desse grande historiador o fundador da antrolopologia histórica.\footnote{Foi isso que Georges Duby reconheceu em seu prefácio à 7ª edicção de \textit{Apologie pour l'histoire ou Métier d'historien}~(1974): "Quando aos 56 anos, nas últimas linhas que escreveu, o Bloch da Resistência af{\kern0pt}irma mais uma vez que as condições sociais são, 'em sua natureza produnda, mentais' ~(p. 158), não está ele nos convocando a retomar seu primeiro, seu verdadeiro grande livro, a reler \textit{Os reis taumaturgos} e a prosseguir essa história das mendalidades que ele abandonara, mar da qual o jovem Bloch, já precisamente cinquanta anos, foi talvez o inventor?" (p. 15).} 

~\\ \large \textit{GÊNESE DE "OS REIS TAUMATURGOS} ~\\

No estado atual de nossos conhecimentos sobre Marc Bloch, e esperando que o que se conservou de suas cartas e das de seus correspondentes talvez nos traga precisões, se não revelações, pode-se dizer que a gestação de \textit{Os reis taumaturgos} estendeu-se por uma dúzia de anos e benef{\kern0pt}iciou-se de três experiências principais, duas de ordem intelectual e no intervalo entre estas uma de ordem existencial.\footnote{Em primeiro lugar agradeço a Etienne Bloch, f{\kern0pt}ilho de Marc Bloch, ter colocado a meu dispor as informações e os documentos a respeito de \textit{Os reis taumaturgos} e ter-me autorizado a trabalhar sobre o fundo de papéis de Marc Bloch depositados nos Archives Nationales, os quais pude, graças à amabilidade da sra. Suzanne d'Huart, conservadora-chefe, consultar as melhores condições. Esse fundo traz o código AB XIX 3796-3852 (o código AB XIX deigna a documentação dos grandes eruditos depositadas nos Archives Nationales). A maior parte das citações deste Prefácio que não têm referências provém desse fundo. Também agradeço a meu amigo André Burguière diversas indicações preciosas.}

A primeira tem por teatro a Fondation Thiers, em Paris, onde Marc Bloch ~(que em 1908 saíra da École Normale Supérieure como professor agrégé de história) foi pensionista de 1909 a 1912. Depois vem a experiência da guerra de 1914-18, que ele terminou capitão, após ter sido citado quatro vezes por bravura e ter recebido a Croix de Guerre.

Enf{\kern0pt}im, deve-se considerar a atmosfera da faculdade de letras da Universidade de Estrasburgo, em que foi nomeado \textit{chargé de cours} em dezembro de 1919 e \textit{professeur} em 1921.

A atividade científ{\kern0pt}ica de Marc Bloch começa em 1911-2. Ele publica seus primeiros artigos. Até a guerra, esses estudos testemunham três centros de interesse, claramente ligados entre si. De início, a história institucional do feudalismo medieval, sobretudo o lugar da realeza e o da sevidão no sistema feudal, primeiros passos de um estudo que depois da guerra será paralisado (em virtude das disposições tomadas em favor dos universitários ex-combatentes) num embrião de tese? "Rois et serfs - un chapitre d'histoire capétienne" [Reis e servos - um capítulo de história capetíngia]. Em seguida (no quadro de geograf{\kern0pt}ia histórica que teve, a partir de Vidal de la Blanche e dos sucessores deste, inf{\kern0pt}luência tão grande sobre a nova escola histórica francesa do período entre as duas guerras), uma região: a Île-de-France. Enf{\kern0pt}im, uma primeira dissertação sobre o método: a pouquíssimo conhecida preleção pronunciada na distribuição dos prêmios do liceu de Amiens em 1914, às vesperas da Grande Guerra: "Critique historique et critique du témoignange" [Crítica histórica e crítica do testemunho].

\begin{sloppypar}Entre esses primeiros ensaios, um, que apareceu em 1912, merece atenção especial: "Les formes de la rupture de l'hommage dasns l'ancien droit féodal"\footnote{Publicado na \textit{Nouvelle Revue Historique du Droit Français et Etranger}, t, XXXVI, mars-avril 1912, pp. 141-77, e reeditado em Marc Bloch, \textit{Mélanges historiques}, Paris, 1963 (Bibliothèque Générale de l'École Pratique des Hautes Études, vi\textsuperscript e \textnormal section, sepven), t. I, pp. 189-209.)} [Formas da ruptura da homenagem no antigo direito feudal], Marc Bloch descreve ali um "rito" feudal: o "arremesso da palha" e, às vezes, a "ruptura" da palha (\textit{exfestucatio}), signif{\kern0pt}icando, realizando a ruptura da homenagem. Interesse precoce, portanto, pelo ritual nas instituições do passado; e, ante a indiferença da quase totalidade dos historiadores em geral e dos historiadores do direito medieval francês em particular (duas notas de Gaston Paris e uma alusão de Jacques Flach), Marc Bloch volta-se para os historiadores alemães do direito medieval, então abertos à etnograf{\kern0pt}ia e ao comparativismo; um artigo de Ernst von Moeller e, sobretudo, "o grande trabalho do sr. Karl von Amira", \textit{Der Stab in der germanischen Rechtssynbolik}\footnote{Referências precisas sobre esses dois trabalhos encontram-se no citado artigo de Marc Bloch, \textit{Mélanges historiques,} I, p. 190, n. 2.} [O báculo no simbolismo legal germânico].\end{sloppypar}

~\\ \large \textit{O TRIO DA FONDATION THIERS} ~\\

Onde está então Marc Bloch? Depois de diversas passagens por universidades em 1908-9, em Berlim e em Leipzig, ele termina sua permanência na Fondation Thiers. Reencontrou ali dois antigos companheiros da École Normale, Lois Gernet, o helenista (formado em 1902), e Marcel Granet, o sinólogo (da turma de 1904, como Marc Bloch). Os três jovens eruditos organizaram entre si um pequeno grupo de pesquisas. Parece que a inf{\kern0pt}luência de Granet sobre seus dois amigos foi particularmente importante. A problemática e os métodos daquele que iria renovar a sinologia contribuíram a orientar Louis Gernet e Marc Bloch para percepções mais amplas que as da historiograf{\kern0pt}ia tradicional acerca da Grécia antiga e do Ocidente medieval. Antes que \textit{Os reis taumaturgos} apareça em 1924, Marcel Granet terá publicado \textit{Fêtes et chansons anciennes de la Chine} [Festas e canções antigas da China] (1919) e \textit{La religion des chinois} [A religião dos chineses] (1922) e  iniciado a ref{\kern0pt}lexão e as pesquisas que o conduzirão às duas grandes sínteses: \textit{La civilization chinoise} [A civilização chinesa] (1929) e \textit{La penseé chinoise} [O pensamento chinês] (1934). Escreveu também \textit{La féodalite chinoise} [O feudalismo chinês], publicado em 1932 em Oslo, onde também viera a lume no ano anterior \textit{Les caractères originaux de l'histoire rurale française} de Marc Bloch, a quem Granet seguira à capital norueguesa como convidado estrangeiro do Instituto para o Estudo Comparado das Civilizações (apresentado por Marc Bloch nos \textit{Annales} em 1930, pp. 83-5). Desde suas primeiras fases, a obra de Granet contribuiu para conf{\kern0pt}irmar o interesse de Marc Bloch pelos ritos e mitos, pelas cerimônias e lendas, pela psicologia coletiva comparada, pelos "sistemas de pensamento" e de crença das sociedades do passado\footnote{Como estudo (anterior a \textit{Os reis taumaturgos}) de um rito jurídico por Marcel Granet, ver "Le dépot de l'enfant sur le sol", publicado em 1922 em \textit{La Revue Archéologique}.}

Lois Gernet, cujo ensinamento f{\kern0pt}icou em seguida restrito, por muito tempo, à Universidade de Argel (verdade é que ele acolheu ali um jovem historiador chamado Fernand Braudel) e cuja obra foi escandalosamente marginalizada pelo helenismo universitário reinante, não está menos próximo de Marc Bloch por seu pensamento e por seu comportamento. A partir de 1917, Gernet publicou suas \textit{Recherches sur le développement de la pensée juridique et morale en Grèce} [Pesquisas sobre o desenvolvimento do pensamento jurídico e moral na Grécia]. Sua grande síntese, \textit{Le génie grec dans la religion} [O espírito grego na religião], escrita com André Boulanger para o período helenístico, é publicada em 1932 - mas sua notoriedade data apenas de sua reedição em 1970, quando a compilação póstuma de seus artigos, \textit{Anthopologie de la Grèce antique} {Antropologia da Grécia antiga} (1968, reeditada em 1982), permite enf{\kern0pt}im avaliar sua envergadura e compreender sua inf{\kern0pt}luência sobre a grande escola francesa contemporânea da antropologia histórica da Grécia antiga (Jean-Pierre Vernant; Pierr Vidal-Naquet; Marcel Detienne, vindo de Liège; Nicole Loraux; François Hartog; e outros). As discussões de Marc Bloch (e de Granet) com Gernet só f{\kern0pt}izeram aprofundar sua atenção para o etnolegalismo, o mito, o ritual, o comparatismo perspicaz e prudente.\footnote{Dessas informações sobre o grupo Bloc-Gernet-Granet na Fondation Thiers em 1909-12, devo o essencial a Ricardo Di Donato, professor da Scuola Normale Supeiore di Pisa, o qual prepara um grande trabalho sobre Louis Gernet e a quem agradeço calorosamente.}

~\\ \large \textit{A GRANDE GUERRA} ~\\

Depois, vem a segunda experiência: a guera de 1914-18. Para Marc Bloch, foi uma aventura extraordinária. As memórias que escreveu durante o primeiro ano do conf{\kern0pt}lito mostram aliar com simplicidade um patriotismo ardente, um desejo de nada esconder a respeito das realidades sórdidas e cruéis da vida dos combatentes. Mas conserva sempre uma lucidez que lhe permite, mesmo durante a ação mais acirrada, contemplar com desprendimento a ação, lançar um olhar repleto de humanidade (ainda que sem condescendência) sobre os homens a seu redor e sobre si mesmo. Esforça-se constantemente para, como historiador, ref{\kern0pt}letir sobre o que vê e sobre o que vive. No primeiro dia em que é lançado na batalha, 10 de setembro de 1914, observa que: "O senso de curiosidade, o qual raramente me abandona, não me deixara". À \textit{curiosidade}, primeiro estímulo da história, junta-se em seguida um trabalho de pesquisa da \textit{memória}. Sempre anota num caderno os acontecimentos do dia, até que, depois de 15 de novembro de 1914, um ferimento e a doença o impeçam de continuar esse registro de marcha. No início de 1915, quando uma doença grave faz que seja enviado para a retaguarda e obriga-o a um período de repouso para convalescença, ele apressa-se a escrever suas lembraças. Não quer f{\kern0pt}icar na dependência da memória: Esta opera no passado "uma triagem que frequentemente me parece pouco judiciosa". Ao f{\kern0pt}inal dessas recordações dos cinco primeiros meses de guerra, tira, na qualidade de historiador, as conclusões de sua experiência de combatente. Esboça os temas que retomará em 1940 en \textit{L'étrange défaite}\footnote{\textit{L'étrante défaite}, publicação póstima, Paris, 1946 (nova edição está sendo preparada pela Éditions Gallimard).} [A estranha derrata]. Mas para Marc Bloch o essencial é aquilo que concerne à psicologia, psicologia individual dos soldados e dos of{\kern0pt}iciais, psicologia coletiva dos grupos de guerreiros.\footnote{Ver Marc Bloc, "Souvenirs de guerre 1914-1915", \textit{Cahiers des Annales}, 26, Paris, 1969. Quando, como of{\kern0pt}icial, precisou garantir a defesa de homens levados a conselho de guerra, Marc Bloch pôde enriquecer sua experiência da psicologia do soldado. Dessas defesas se conservaram algumas notas. Ver o catálogo da Exposition Marc Bloch (preparado por André Burguière e Claude Chandonnay), École des Hautes Études en Sciences Sociales, mai 1979.}

Com muita perspicácia e sutileza, carlo Ginzburg revelou e analisou a maneira pela qual \textit{Os Reis Taumaturgos} nasceram da experiência da guerra de 1014-8. Marc Bloch viu ali a reconstrução de uma sociedade quase medieval, uma regrassão a uma mentalidade "bárbara e irracional". A propagação de notícias falsas, segundo ele a principal forma dessa regrassão, inspirou-lhe um de seus artigos mais notáveis: "Rèf{\kern0pt}lexions d'un historien sur les fausses nouvelles de la guerre"\footnote{\textit{Revue de Synthèse Historique}, t. 33, 1921, pp. 13-35, retomado em \textit{Mélanges historiques}, t. 1, Paris, 1963, pp. 41-57.} [Ref{\kern0pt}lexões de um historiador sobre as falsas notícias da guerra]. Sobretudo, mostra como a censura, desacreditando os materiais escritos que foram submetidos a seu exame repressivo, acarretou "um renovamento prodigioso da tradição oral, mãe antiga das lendas e dos mitos". Desse modo, a guerra oferece ao historiador um inesperado meio de observar diretamente o passado medieval: "Por um golpe ousado, que o mais audacioso dos experimentadores jamais se atreveu a sonhar, a censura, abolindo os séculos decorridos, reconduz o soldado do \textit{front} aos meios de informação e ao estado de espírito das idades antigas, antes do jornal, antes da folha de notícias impressa, antes do livro". Mas o ceticismo que o historiador adquire em face da difusão das falsas notícias não atinge a "história jurídica, econômica ou religiosa" nem, ainda menos, a história da psicologia coletiva: "Aquilo que há de mais profundo em história poderia ser também aquilo que há de mais seguro". Assim serão \textit{Os reis taumaturgos}, um mergulho na história "profunda".\footnote{Carlo Ginzburg, prefácio à tradução italiana \textit{I re taumaturghi}, Turim, Einaudi, 1973, pp. XI-XIX.}

Daí o diagnóstico que, no f{\kern0pt}inal da obra, Marc Bloch dará cerca do milagre régio: uma "gigantesca notícia falsa". Expressão que retomará em 1932 para def{\kern0pt}inir o fenômeno estudado por Georges Lefebvre em outro grande livro de história das mentalidades: \textit{La Grande Peur de 1789}\footnote{Comentário da obra de G. Lefebvre por Marc Bloch, "L'erreur collective de la 'grande peur' comme symptôme d'un etat social", em \textit{Annales d'Histoire Economique et Sociale}, v, 1933, pp. 301-4.}[O Grande Medo de 1789].

A experiência da guerra reforçou em Marc Bloch a convicção de que, se "a imcompreensão do presente nasce fatalmente da ignorância do passado", não é menos verdadeiro que se faz preciso "compreender o passado pelo presente", como relembrará em \textit{Métier d'historien}. Donde a importância que ele atribui ao "método regrassivo". A psicologia dos soldados e dos homens de 1914-8 escrlarecerá a atitude das gentes da Idade Média (até o século XVIII) para com o milagre régio.

Em todo o caso, o projeto da pesquisa que iria terminar na redação de \textit{Os reis taumaturgos} concretizou-se no espírito do jovem historiador durante a Grande Guerra. Seu colega Charles-Edmond Perrin revelou que em fevereiro de 1919, durante excursão que f{\kern0pt}izeram juntos aos Vosges, quando ainda não haviam sido desmobilizados, Marc Bloch disse-lhe: "Quando eu tiver concluído meus rurais, abordarei o estudo da unção na sagração real de Reims".\footnote{Prefácio de Ch.-Éd. Perrin a Marc Bloch, \textit{Mélanges historiques. op.cit., p.xi.}.}

~\\ \large \textit{ESTRASBURGO} ~\\

Insistirei menos (porque é mais bem conhecida) na terceira circunstância que favoreceu a escolha def{\kern0pt}initiva e a redação do estudo sobre o milagre régio: o ambiente da Universidade de Estrasburgo, da qual Marc Bloch foi nomeado \textit{maître de conférences} em outubro de 1919.\footnote{Ver Lucien Febvre, "Souvenirs d'une grande histoire: Marc Bloch et Strasbourg", em \textit{Mémorial des années 1939-1945}, Estrasburgo, Faculté des Lettres: retomado em \textit{Combats pour l'histoire}, Paris, A. Colin, 1953.} Assim que a guerra terminou, a Universidade de Estrasburgo, cidade que voltara a ser francesa, recebia dos poderes públicos uma atenção especial, destinada a eclipsar a lembrança da universidade alemã e a fazer daquela instituição reencontrada uma vitrine intelectual e científ{\kern0pt}ica da França ante o mundo germânico. Jovens mestres muito brilhantes foram nomeados para lá: o historiador Lucien Febvre (nascido em 1878), o qual é necessário nomear antes de todos porque ali se deu o encontro decisivo que resultaria na fundação conjunta dos \textit{Annales d'Histoire Économique et Social} em 1929; e outros, como o especialista em Antiguidade romana André Piganiol, o medievalista Charles-Edmond Perrine, sobretudo, o grande historiador da Revolução Francesa Georges Lefebvre. Mas também o fundador da sociologia religiosa na França, Gabriel le Bras; o geógrafo Henri Baulig; o f{\kern0pt}ilósofo Ernest Hoepf{\kern0pt}fner; e, principalmente, o médico e psicólogo Harles Blondel e o sociólogo Maurice Halbwachs. Blondel já publicou em 1914 \textit{La conscience morbide} [A consciência mórbida], publicará em 1926 \textit{La mentalité primitive} [A mentalidade primitiva] e trabalha em seu grande livro, \textit{Introcuction à la prychologie collective} [Instrodução à psicologia coletiva] (1928), que Marc Bloch comentará na \textit{Revue Historique} em 1929. Como destacou Georges Duby, Blondel provocava os historiadores -- mas isso acontecia quatro anos depois de \textit{Os reis taumaturgos}!
-- af{\kern0pt}irmando que "não era o caso de obstinar-se em determinar de imediato as maneiras universais de sentir, de pensar e de atir". Chamamento para uma história diferencial (no tempo e no espaço) das mentalidades e das condutas. Um ano após \textit{Os reis taumaturgos}, Maurice Halbwachs trazia a lume um livro capital para todo o domínio do que hoje denominamos ciências humanas e sociais: \textit{Les cadres sociaux de la mémoire} [Os esquemas sociais da memória]. No ano de sua publicação, Marc Bloch consagrou-lhe londo artogo na \textit{Revue de Synthèse Historique} de Henry Berr, o pioneiro na renovação da história e das ciências humanas.\footnote{"Mémoire collective, tradition et coutume à propos d'un livre récent", \textit{Revue de Synthèse Historique,} t. 40, 1925, pp. 73-83.} Memória e sociedade, portanto memória e história -- qual tema teria mais condição de seduzir Marc Bloch?

Em Blondel e em Halbwachs, Marc Bloch reecontra os alunos do sábio que mais marcou sua formação intelectual, o sociólogo Émile Durkheim, que morreu em 1917. O Durkheim que em 1912 publicou, com base no estudo do sistema totêmico australiano, \textit{Les formes élémentaires de la vie religieuse} [As formas elementares da vida relif{\kern0pt}iosa], em que o sarfado é def{\kern0pt}inido "como uma representação da sociedade".\footnote{J.-L. Fabiani, artigo "Durkheim (Emile)", em \textit{La nouvelle histoire}, J. Le Gof{\kern0pt}f, R. Chartier e J. Revel (eds.), Paris, 1978, p. 149.} O Durkheim cuja inf{\kern0pt}luência sobre Marc Bloch é, com muita perspicácia, def{\kern0pt}inida numa carta que Henri Sée lhe envia para agradecer-lhe e felicitá-lo por \textit{Os reis taumaturgos}, na qual fala também do comentário que Marc Bloch acaba de publicar na \textit{Revue Historique} sobre o livro de Lucien Febvre (com a colaboração de Lionel Baraillon), \textit{La terre et l'évolution humaine, Introduction géographique à l'histoire} [A terra e a evolução humana. Introdução geográf{\kern0pt}ica à história] (1922): "Suas observações acerca do livro de Lucien Febvre", escreve Henri Sée, "parecem-me muito justas. No fundo, a história [...]
deve aproximar-se bem mais da sociologia que da geograf{\kern0pt}ia; e o método sociológico, tal como o def{\kern0pt}iniu Durkheim, é em grande parte um método histórico. De fato, em 1898 Durkheim, colocou no frontispício do primeiro número do \textit{Année Sociologique} uma referência a Fustel de Coulanges, o mesmo Fustel de Coulanges a que Marc Bloch recorre com frequencia no \textit{Métier d'historien} e que foi seu inspirador desde a juvetude. Em 1909, Christian Pf{\kern0pt}ister na carta em que recomenda à Fondation Thiers a candidatura de Marc Bloch, escreve que este ocupa-se com questões de história social negligenciadas desde Fustel de Coulanges.\footnote{Ver Carole Fink, op. cit., n. 9.} Marc Bloch expressou no \textit{Métier d'historien} as dívidas que ele mesmo e os historiadores desejosos de escapar à mesmice da história positivista universitária tinham para com Durkheim e sua escola: "A esse grande esforço nossos estudos devem muitíssimo. Ele ensinou-nos a analisar com maior profundidade, a considerar os problemas mais de perto, a pensar, eu ousaria dizer, menos barato".\footnote{\textit{Apologie pour l'histoire ou Métier d'historien}, 7ª ed., 1974, p. 27. Sobre Durkheim, a história e Marc Bloch, ver Robert N. Bellah: "Durkheim and history", em \textit{American Sociological Review}, 24, 1959, pp. 447-61, e R. Colbert Rhodes, "Emile Durkheim and the historical thought of Marc Bloch", em \textit{Theory and Society}, 6, n. 1, 1978, pp. 45-73.}

Portanto, Estrasburgo foi para Marc Bloch o contato vivo, por intermédio dos colegas e amigos, com as ciências sociais, irmãs da história. \textit{Os reis taumaturgos} cresceram nesse humo interdisciplinar. Não surpreende que reencontremos as ciências sociais na obra, como Henri Sée soube tão bem manifestar: "Seu último capítulo, escelente, não interessará apenas aos historiadores. folcloristas, psicólogos e sociólogos ali encontrarão matéria para ref{\kern0pt}lexões".

Para concluir esta evocação do que \textit{Os reis taumaturgos} devem à Estrasburgo dos anos 1919-24, é necessário também pensar na notável biblioteca universitária que alemães (de 1871 a 1918) e franceses, num desaf{\kern0pt}io, tornaram uma questão de honra dotar ricamente. Lucien Febvre disse-o bem: "O pano de fundo era a Bibliothèque, a admirável Bibliothèque Nationale et Universitaire de Strasbourg, com seus tesouros expostos sob nossos olhos, ao alcance de nossas mãos, um instrumento de trabalho incomparável, único na França. Se alguns de nós devem deixar uma obra atrás de si, eles o devem em parte à Bibliothèque. A seus prodigiosos recursos que eles não f{\kern0pt}izeram senão explorar.\footnote{Em "Marc Bloch et Strasbourg", retomado em \textit{Combats pour l'histoire}, p. 400. É claro, Marc Bloch também explorou os recursos da Bibliothèque Nationale em Paris e da British Library em Londres e escreveu inúmeras cartas a arquivistas na França e no estrangeiro.}

~\\ \large \textit{OS MEDIEVALISTAS ALEMÃES} ~\\

Por f{\kern0pt}im, dois fatores muito diferentes devem ter contribuído para impelir Marc Bloch ao estudo do mal régio. O primeiro é sua familiaridade com a obra dos medievalistas alemães e a sedução exercida sobre ela pela erudição alemã e pela problemática germânica. Christian Pfster faz alusões a isso em sua já mencionada carta de 1909. As estadas de Marc Bloch em Berlim e em Leipzig, em 1908-9, trouxeram seus frutos. Uma de suas primeiras exposições na \textit{Revue Hostorique}, em 1921, é consagrada ao livro de Fritz Kern, saído em 1914, \textit{Gottesgnadentum und Widerstandsrecht im früheren Mittelalter. Zur Entwicklungsgeschichte der Monarchie} [Graça de Deus e direito de resistência na alta Idade Média. Contribuição à história do desenvolvimento da monarquia].

Aliás, no dossiê de \textit{Os reis taumaturgos}, entre os papéis de Marc Bloch nos Archives Nationales, há uma carta de agradecimento de Fritz Kern pelo comentário de Marc Bloch. O historiador alemão está profundamente comovido pela atenção que Marc Bloch deu a sua obra e pela delicadeza com que o novo estrasburguês encaminhou-lhe sua recensão. Kern diz que, depois da guerra, só recebe de seus colegas franceses indiferença ou respostas grosseiras. Por isso, esta ainda mais tocado.

Os trabalhos alemães, se não inspiraram, pelo menos já serviram Marc Bloch. Em seu artigo de 1912 sobre a "ruptura da homenagem", utilizara representantes alemães de uma disciplina muito negligenciada na França, o etnolegalismo.\footnote{Ver n. 5.} A historiograf{\kern0pt}ia alemã informa-o e o estimula a ultrapassá-la, em direção à história da autoridade monárquica, das imagens e das insígnias, na qual mais tarde adquirirão celebridade P. E. Schramm e sua escola.

~\\ \large \textit{O IRMÃO MÉDICO} ~\\

Enf{\kern0pt}im, suas afetuosas relações com um irmão médico o levaram a aprofundar o lado clínico de seu tema e os aspectos relativos à medicina popular. No prefácio de \textit{Os reis taumaturgos}, num acréscimo com data de 28 de dezembro de 1923, Marc Bloch lembra a inf{\kern0pt}luência desse irmão, o qual, assim como o pai, morrera antes de ver terminada e publicada a obra em que tivera uma participação que Marc Bloch declara decisiva.\footnote{Ver aqui, pp. 22-23. Revisar nesta nota onde aparece a seção Ritos do prefácio.}

~\\ \large \textit{HISTÓRIA TOTAL DE UM MILAGRE} ~\\

Agora, é preciso olhar \textit{Os reis taumaturgos} como Marc Bloch o concebeu e escreveu, esforçando-nos por reinserir o livro no pensamento histórico e antropológico de sua época, o início dos anos 20.
O que Marc Bloch quis foi fazer a história de um milagre e, simultaneamente, a da crença nesse milagre. Aliás, as duas confundem-se em parte. Marc Bloch mostrou que o milagre existe a partir do momento em que se \textit{pode} (não ha determinismo em Marc Bloch, mas correlações racionais entre os fenômenos históricos, sem que se tenha a identif{\kern0pt}icação hegeliana entre o racional e o real) acreditar nele; e que o milagre declina e então desaparece a partir do momento em que não se pode mais acreditar nele. "Se eu não receasse tornar ainda mais pesado um título já demasiado longo,\footnote{\textit{Les rois thaumaturges. Étude sur le caractère surnatural attribué à la puissance royale particulièrement en France et en Angleterre}.} teria dado a este livro mais um subtítulo: \textit{História de um milagre"} (p. 45 Introdução (conferir a página correta no arquivo f{\kern0pt}inal)).

~\\ \large \textit{A LONGA DURAÇÃO} ~\\

Esse milagre, ele quer "explicá-lo em sua duração e em sua evolução", no centro de uma "explicação total". Reconhecemos aqui dois dos grandes temas da "escola" dos \textit{Annales: a história global ou total}("explicação total" é tão melhor! É claro, isso permanece um ideal, um caso-limite, um horizonte mais ou menos inacessível) e a \textit{longa duração}, da qual Fernand Braudel, em artigo justif{\kern0pt}icadamente célebre, devia em 1958 explicar sua def{\kern0pt}inição,\footnote{\textit{F. Braudel, "Histoire et sciences sociales, La longue durée", Annales E. S. C.,} 1958, pp. 725-53, retomado em \textit{Écrits sur l'histoire}, Paris, Flammarion, 1969, pp. 41-83.} depois dela ter dado a mais bela ilustração em \textit{La Méditerranée et le monde méditerranéen à l'époque de Philippe II} [O Mediterrâneo e o mundo mediterrânico na época de Filipe II] (1949). A longa duração não é forçosamente um longo período cronológico; é aquela parte da história, a das estruturas, que evolui e muda o mais lentamente. A longa duração é um ritmo lento. Pode-se descobri-la e observá-la por um lapso de tempo relativamente curto, mas subjacente à história dos eventos e à conjuntura de médio prazo. O pior seria acreditar que a problemática "das origens a nossos dias", raramente compatível com uma problemática histórica científ{\kern0pt}ica, é a longa duração perfeita. Mas o caso de \textit{Os reis taumaturgos} -- em que o historiador tem a sorte de conhecer o começo e o f{\kern0pt}im de um fenômeno histórico, de poder estudá-lo durante toda a sua vida histórica, desde seu nascimento e sua gênese até seu declínio e sua desaparição -- é uma oportunidade excepcional. Assim, Marc Bloch pode af{\kern0pt}irmar que o milagre régio, o rito do toque, "nasceu na França em torno do ano 1000, na Inglaterra cerca de um século mais tarde", e que tal rito desapareceu na Inglaterra com a chegada da dinastia Hanover em 1714, na França a 31 de maio de 1825, quando Carlos X, após sua sagração (29 de maio), foi o último francês a tocar os escrofulosos.

~\\ \large \textit{O CULTO ÀS ORIGENS} ~\\

Por um dúplice paradoxo, a parte de \textit{Os reis taumaturgos} que hoje deve ser revisada é a concernente às \textit{origens} do toque régio. Paradoxo porque Marc Bloch, o qual bem cedo denunciará o que chamará em \textit{Métier d'historien} o "culto às origens" [idole des origines], aqui presta sacrifício a esse conceito, que conduz à confusão entre origens, fontes (outra palavra perigosa, como se em história as coisas surgissem sem esforço*{\let\thefootnote\relax\footnote{(*)Há aí um jogo de palavras intraduzível, entre \textit{sources} ("fontes") e o idiomatismo \textit{couler de source} ("surgir, produzir-se naturalmente, sem esforço"). (N.T)}} ou por parto natural) e causas. Ora, desde \textit{Os reis taumaturgos} Marc Bloch deixava perceber conceitos mais fecundos: herança, scolha, nascimento, gênese, com a ideia básica de que "um fenômeno histórico jamais se explica plenamente fora do estudo de seu momento". O segundo paradoxo está em que a erudição, sempre necessária, fundamental a um historiador, não é objetivamente mais sólida que as hipóteses, as interpretações, as ideias. Perigosa ilusão essa dos historiadores que pensam que a erudição bem praticada pode fazer chegar a certezas absolutas, a conclusões def{\kern0pt}initivas. Também a erudição -- mesmo a melhor -- é frágil. Surgem outros documentos, modif{\kern0pt}icando o lugar que um documento precedentemente conhecido ocupava na sequência cronológica. Um ponto de vista dá novos sentidos ao documento antigo, inclusive no âmbito da literalidade e da historicidade. Graças às descobertas, às novas técnicas, o passado tem, desde o nível da documentação, um belo futuro diante de si. Sejamos então, desde a fase erudita do trabalho histórico, bastantes modestos, humildes em face tando do futuro quanto do passado.

O caso de \textit{Os reis taumaturgos} parece-me exemplar. Em seguida a uma coleta e a uma crítica de documentos que mesmo os eruditos menos seduzidos pelo método de Marc Bloch elogiaram por atender às exigências científ{\kern0pt}icas mais rigorosas, ele isola do lote de documentos um texto. É uma carta de um clérigo de origem francesa, Pierre de Blois, que vive na corte do rei Henrique II da Inglaterra. Ele escreve por volta de 1180: "Confesso que assistir o rei é [para um clérigo] cumprir uma tarefa santa; pois o rei é santo; é o ungido [\textit{christus}] do Senhor, não foi em vão que o rei recebeu o sacramento da unção, cuja ef{\kern0pt}icácia, se por acaso alguém a ignorasse ou a colocasse em dúvida, seria amplamente demonstrada pela desaparição dessa peste que ataca a virilha e pela cura das alporcas".\footnotemark[24]\footnotetext[24]{O texto latino (editado no tomo 207 da \textit{Patrologia Latina de Migne}, col. 440) que Marc Bloch examinou no manuscrito da Bibliothèque Nationale de Paris, Nouvelles acquisitions latines 785. f. 59, e que mandei verif{\kern0pt}icar nos manuscritos mais antigos (pois por um momento pensei que o texto autêntico podia ser não \textit{inguinariae pestis}, a peste inguinal, a peste negra, mas \textit{igniariae pestis}, o mal do fogo, ou seja, o fogo-de-santo-antônio, causado pelo ergotismo do centeio atestado na época), diz: "f{\kern0pt}idem ejus plenissiman faciet defectus inguinariae pestis, et curatio scrophularum" (\textit{Os reis taumaturgos}, pp. 61-2 e p. 62, n, 25).}

Em vista de meu interesse pela história da \textit{peste inguinal}, também chamada de \textit{peste bubônica} ou \textit{peste negra} na Idade Média, f{\kern0pt}iquei intrigado, relendo \textit{Os reis taumaturgos}, por esse texto de Pierre de Blois, que atribuía a Henrique II (o qual morreu em 1189) o mérito de ter feito desaparecer uma epidemia dessa peste inguinal. Sabemos hoje (aqui, pode-se falar em saber adquirido, pois um fenômeno maciço como a peste negra forçosamente constaria bastante abundantes documentos do século XII) que não houve epidemia de peste inguinal no Ocidente entre os século VII e 1347.\footnote{Dr. Jean-Noël Biraben, \textit{Les hommes et la peste en France et dans les pays européens et méditerranéens}, 2 vols., Paris-Haia, 1976. Dr. J.-N. Biraben e J. Le Gof{\kern0pt}f, "La peste dans le haut Moyen Age", em \textit{Annales E. S. C.}, 1960, pp. 1484-508.} Mas há sessenta anos a historiograf{\kern0pt}ia estava na mais perfeita confusão sobre a cronologia da Peste Negra, e a grande maioria dos historiadores sérios, (incluindo o douto e curioso Marc Bloch) pouco a pouco se ia interessando por essa doença que, não sem motivo, eles não encontravam em seus documentos referentes ao período entre o século VII e meados do século XIV. Contudo, Marc Bloch sente certo desconforto diante dessa indicação. Ele dizia a si mesmo: "Não sabemos ao certo a que aludem essas últimas palavras; talvéz a uma epidemia de peste bubônica que, acreditava-se, teria cedido à miraculosa inf{\kern0pt}luência do rei. Um excelente historiador da medicina, o dr. Crawfurd, af{\kern0pt}irma que a confusão entre certas formas de ínguas pestosas e adenite da virilha não era nada impossível para um homem daquela época, Pierre de Blois não era médico" etc. (p. 62). Mas, se Pierre de Blois não era boa testemunha da peste inguinal, por que o seria das escrófulas?

Ora, March Bloch conclui: "Portanto, Henrique II curava os escrofulosos". Se Marc Bloch acreditava poder, com ressalvas, concluir que os textos do século XII permitiam supor que o toque régio existia na Inglaterra desde o começo daquele século, é sem reservas que af{\kern0pt}irma ser esse texto o mais antigo testemunho \textit{seguro} (p. 67) de que o rei da Inglaterra curava as escrófulas.

Consegui determinar a proveniência muito provável da menção de Pierre de Blois ao desaparecimento de uma epidemia de peste ocasionado pelo poder do rei.\footnote{Devo agradecer a Marie-Claire Gasnault, que me ajudou nessa pesquisa.} Na \textit{Historia Francorum} [Historia dos Francos] (X, I), Gregório de Tours conta que o papa Gregório Magno, no ano de sua elevação ao pontif{\kern0pt}icado (590), ordenou ao povo romano litanias -- procissão e cantos de penitência -- para fazer cessar a "peste inguinal", esta uma epidemia bem real, que devastava Roma. Essa litania maior, por oposição à litania menor das Rogações, foi daí em diante celebrada por toda a Cristandade dia 25 de abril e entrou na liturgia regular. Já no começo do século VII, Beda a menciona em sua \textit{Homilia 97, De major litania} (P. L., t. 94, col. 499). Pouco antes do momento em que Pierre de Blois escreve sua carta, o liturgista parisiense Jean Bleth, em sua \textit{Summa de Ecclesiastias of{\kern0pt}f{\kern0pt}iciis}, no capítulo "Das litanias", relembra a origem da grande litania que Gregório Magno instituiu a f{\kern0pt}im de fazer desaparecer uma "pestis inguinaria".\footnote{Jean Beleth, \textit{Summa de ecclesiasticis of{\kern0pt}f{\kern0pt}iciis}, ed. H. Douteil, Turnholt, 1976. "Corpus christianorum, Continuatio medievalis", XLI, pp. 232-4.} No século XIII, o fato ainda é relatado por Iacopo da Varezze na \textit{Legenda aurea} (em torno de 1255); e o dominicano Jean de Mailly, em sua obra inédita \textit{Abbreviatio in gestis et miraculis sanctorum} (em torno de 1243), descreve a \textit{major letania} recordando a origem desta. Ele conta a lenda segundo a qual, depois da litania, Gregório Magno viu no alto de um palácio romano um anjo enxugar sua espada ensanguentada e recolocá-la na bainha, donde o nome Santo Angelo dado ao monumento. Acrescenta que essa procissão é chamada a das "cruzes negras".\footnote{Iacopo da Varezze, \textit{Legenda aurea}, a litania maior e a litania menor, Jean de Mailly, \textit{Abbreviatio in gestis et miraculis sanctorum}, Paris, Bibliothèque Mazarine, ms. 1731, f{\kern0pt}f. 55v-56, cujo conhecimento devo a Marie-Claire Gasnault.} Era realizada no dia de s. Marcos, 25 de abril, e Joinville relembra que s. Luís nasceu nesse dia (1214), o que foi um sinal premonitório de sua trágica morte diante de Túnis.

Pierre de Blois, portanto, apenas reproduziu uma tradição literária e uma prática litúrgica bem conhecidas, as quais subsistiram durante séculos em que não houve nenhuma epidemia de peste negra. Por conseguinte, Henrique II não fez desaparecer peste alguma, mas Pierre de Blois atribui-lhe um milagre de Gregório Magno que permanecia na hagiograf{\kern0pt}ia e na liturgia. No que se refere à cura das escrófulas, não terá Pierre de Blois feito a mesma coisa? Não se pode af{\kern0pt}irmá-lo porque (diferentemente do que verif{\kern0pt}iquei acerca da desaparição da peste negra) não encontrei nenhuma tradição anterior precisa a respeito disso; no entanto, a carta de Pierre de Blois está bem desacreditada no que concerne à historicidade dos milagres de Henrique II.

Ora, sem destacar essa carta, mas analisando com muita atenção outros textos nos quais Marc Bloch baseia sua demonstração de que a origem do toque régio das escrófulas está no começo do século XII na Inglaterra e no século XI na França, um historiador britânico, Frank Barlow, acaba de demonstrar convincentemente que nenhum desses textos permite af{\kern0pt}irmar com segurança que as coisas tenham sido assim. Segundo Barlow, e essa era minha impressão, se no século XII temos para os reis da França uma única menção segura ao toque régio das escrófulas (por Luís VI), não há nenhuma prova de que eles tenham tocado de maneira regular antes de s. Luís. Quanto aos reis da Inglaterra, é preciso fazer recuar a 1276 a primeira menção segura a esse rito.\footnote{Frank Barlow, "The king's evil", em \textit{English Historical Review, 1980, pp. 3-27. O autor desse importante artigo, que rende homenagem à obra pioneira de Marc Bloch, diz (p. 25) que o milagre régio "was not a manifestation of holiness but of regality". Não se trata de santidade e sim de sacrealização, de realeza sagrada.}}

Desse modo, é provável que apenas em meados do século XIII o rito régio da cura das escrófulas se tenha tornado uma prática habitual na França e na Inglaterra. Mas a essência da demonstração de Marc Bloch mantém-se intacta. No decorrer da Idade Média, graças a um conjunto de ritos e a uma crença especial, dois reis cristãos tornaram-se personagens sagradas, curandeiros milagrosos. É uma variante cristã da realeza sagrada. Deus, ao lado dos santos, escolhe reis de duas nações para operar milagres em Seu nome. A Igreja deve conceder esse novo poder régio, ainda que o controlando. Só que esses reis levaram mais tempo para adquirir esse poder do que Marc Bloch presumiu. Talvez o contexto do século XIII (posição dos leigos, evolução dos ritos e dos gestos, concepção da santidade, atitudes para o corpo e a doença etc.), muito mais que seus aspectos propriamente políticos, esclareça o milagre régio com mais exatidão do que Marc Bloch pôde fazê-lo situando seu estabelecimento mais cedo.

~\\ \large \textit{UNÇÃO E POLÍTICA} ~\\

Pesquisando as "origens" {--} quer dizer, o começo cronológico do milagre régio --, Marc Bloch já encontra os dois temas essenciais de sua obra: o vínculo entre o poder taumatúrgico e a \textit{sagração} (ou, mais precisamente, a \textit{unção}); e as políticas desse recurso ao sagrado.

Como indicam os manuais litúrgicos da sagração dos reis da França (as o\textit{ordines} da sagração) no século XIII,\footnote{Sobre os \textit{ordines} da sagração dos reis da França, o estudo clássico é o de P, E. Schramm, "Ordines -- Studien II: Die Kronung bei den westfranken und den Franzosen", em \textit{Archiv für Urkunden Froschung}, 15, 1938, pp. 3-55. Mas nele fervilham erros causados por mau conhecimento e má datação dos manuscritos. Importantes e juduciosas retif{\kern0pt}icações foram feitas por Hervé Pinoteau, "La tenue du sacre de saint Louis IX, roi de France", em \textit{Itinéraires}, nº 162, avril 1972, pp. 120-66, e por Richard A. Jackson, "Les manuscrits des \textit{ordines} du couronnement de Charles V, roi de France, em \textit{Le Moyen Age}, 1976, pp. 67-88. O grupo de antropologia histórica do Ocidente Medieval, da École des Hautes Études en Sciences Sociales, espera (no quadro do estudo e da edição de um dos mais interessantes manuscritos de um desses \textit{ordines}, o ms. lat. 1246 da Bibliothèque Nationale de Paris) propor nova classif{\kern0pt}icação e cronologia dos \textit{ordines} régios franceses do século XIII.} a cerimônia de Reims compreende dois aspectos que são também duas fases sucessivas da cerimônia: a consagração ou unção e a coroação. Da unção os reis da França devivam seu poder miraculoso. No f{\kern0pt}inal da Idade Média, o monarca francês será chamado o rei cristianíssimo, colocando-o acima dos outros reis da cristandade -- e isso porque o óleo com o qual se unge na sagração é o único a ter origem sobrenatural. Esse óleo provém da Santa Âmbula e foi trazido por uma pomba (o Espírito Santo ou sua mensageira) para o batismo de Clóvis por s. Remígio [Rémi]. O rei da França é o único a ser ungido com um óleo divino, vindo do céu (toma-se bastante cuidado para que a rainha seja ungida apenas com um óleo natural). No século XIV, contudo, a monarquia inglesa reivindicará o mesmo privilégio. Em 1318, um dominicano inglês, frei Nicholas of Straton, vai a Avignon explicar ao papa João XXII que o famoso arcebispo Thomas Becket (canonizado em 1173, três anos depois de morrer), enquanto estava exilado na França, recebeu da Virgem uma âmbula destinada a ungir o quinto rei da Inglaterra depois de Henrique II (ou seja, o monarca reinante em 1318, Eduardo II) porque, ao contrário de seu antepassado que mandara matar Becket, esse quinto rei seria um "homem de valor, defensor da Igreja", e desejaria "reconquistar a Terra Santa ao domínio da gente pagã". João XXII nem desdenha nem reconhece of{\kern0pt}icialmente essa história. Mas, pelo menos na Inglaterra, instala-se a convicção de que o rei inglês também era ungido com um óleo sobrenatural.

Ao mesmo tempo, Marc Bloch observava nessa gênese do toque régio o clima político que aí se af{\kern0pt}irmava desde o início do jogo. Política dos reis para com a Igreja, mas também política dos reis ingleses e franceses em seus respectivos reinos em face um do outro. Na França e na Inglaterra, a conquista de um poder miraculoso vai a par com a af{\kern0pt}irmação do poder monárquico confrontado com os grandes senhores feudais, os barões. É um instrumento dinástico. Marc Bloch via aí um dos meios pelos quais os dois reis adquiriam um poder dominante, diferente do poder da hierarquia feudal. Se é preciso deslocar dos séculos XI-XII para o século XIII a aquisição desse poder, trata-se mais de uma consagração que de um meio de consegui-lo.

Mas é também o lance inicial de uma luta de prestígio entre as duas monarquias -- mais precisamente, entre os capetíngios e os Plantagenet. O milagre régio é um dos sinais e um dos objetos de emulação e de concorrência na grande rivalidade franco-inglesa da Idade Média.\footnote{Marc Bloch tendia a acreditar que as iniciativas e os primeiros sucessos tinham vindo da monarquia francesa, imitada pela monarquia inglesa. As regif{\kern0pt}icações cronológicas propostas por Frank Barlow e por mim mesmo não colocam em discussão esse modelo, embora a desloquem no tempo. Como bem havia visto Marc Bloch, é preciso lembrar que, a respeito da sagração régia, as relações entre França e Inglaterra remontam à alta Idade Média: "Os ritos francos e angle-saxões desenvolveram-se paralelamente, e pode-se acreditar que com inf{\kern0pt}luências recíprocas" (\textit{Os reis taumaturgos}, p. 299). Nos \textit{ordines} franceses do século XIII, encontram-se referências precisas aos ritos anglo-saxões.}

~\\ \large \textit{A POPULARIDADE DO MILAGRE} ~\\

Depois das "origens", Marc Bloch chega (entrando visivelmente no que mais lhe interessa) ao problema da "popularidade". Para ele, esse termo designa dois fenômenos que não se correspondem inteiramente. De um lado, está a difusão do milagre -- donde o estudo da frequência dos toques, do número de participantes, da orígem geográf{\kern0pt}ica dos doentes tocados. Nisso, o documento essencial são as contas régias. Infelizmente, no que se refere à França, o incêndio do depósito da Chambre de Comptes no Palais de la Cité, em  1737, deixou apenas fragmentos de registros. Tais fragmentos permitem a Marc Bloch esboçar uma das primeiras verif{\kern0pt}icações quantitativas de uma prática ritual, de um fenômeno de mentalidade. Ele introduziu nesse domínio uma preocupação estatística.\footnote{"[...] as estatísticas do toque merecem o interesse do historiador que procura retratar em nuanças a evolução do lealismo monárquico" (\textit{Os reis taumaturgos}, p.99).}

Mas \textit{popularidade} é também a maneira pela qual o milagre é "recebido" pelo "povo", Por isso, uma história da "recepção" de um fenômeno histórico é esboçada por Marc Bloch numa perspectiva sociopsicológica que conhecemos hoje, como se sabe, grande sucesso, mormente no campo da história literária.\footnote{H. Robert Jauss, \textit{Pour une esthétique de la réception} (trad. do alemão), Paris, Gallimard. 1978.} Coloca um problema essencial para o historiador: como um fenômeno que, sejam quais forem seus alicerces mágicos e folclóricos, foi elaborado pelos meios restritos situados no alto da hierarquia cultural social (o rei e seu círculo, prelados, liturgistas e teólogos) pode atingir e atinge a massa? Essas relações entre teorias e práticas da elite, de um lado, a crença e mentalidade "comuns", do outro, estão no âmago do milagre régio, assim como no de todos os milagres. Eis-nos longe da história das ideias tradicional, de tradição positivista ou idealista (a \textit{Geistesgeschichte} dos mestres alemães), acantonada no céu das ideias e sobre os cimos da sociedade.

É claro, para responder a essa questão Marc Bloch investiga a opinião eclesiástica, a qual tem os privilégios da ideologia of{\kern0pt}icial. Mas, acima de tudo, explota o campo que lhe parece o mais pertinente a essa questão: o da medicina popular, do folclore medical. Ajudado por seu irmão, multiplica pesquisas e leituras. Nessa área, sua bibliograf{\kern0pt}ia é muito rica, e seus papéis testemunham a amplitude de uma curiosidade da qual a obra conservou apenas parte das inquirições e dos resultados. No livro, ele insiste sobretudo na maneira pela qual o toque das escrófulas, justamente quando desaparecem as censuras eclesiásticas a um rito que era suspeito de magia e paganismo, tornou-se "um lugar-comum medical" nos tratados de medicina erudita. Sempre um estimulador de ideias, um indicador de pistas, Marc Bloch aventa a utilidade de fazer um estudo comparado da evolução das ideias medicais e da ideologia religiosa.\footnote{Ver R. Zapperi, \textit{L'uomo incinto. La donna, l'uomo e il potere} (\textit{L'homme enceint. La femme, l'homme et le pouvoir}, trad. francesa no prelo, PUF), Cosenza, 1979, o qual mostra que o mito do homem grávido, instrumento de dominação do homem sobre a mulher, é objeto, no f{\kern0pt}inal da Idade Média e na Renascença, de uma justif{\kern0pt}icativa f{\kern0pt}isiológica "científ{\kern0pt}ica", que ratif{\kern0pt}ica a ideologia religiosa.}

~\\ \large \textit{RITOS} ~\\

Provavelmente, interessa-se ainda mais por uma particularidade do rito inglês, a qual não existe no ritual francês: "o segundo milagre da realiza inglesa, os anéis medicinais". A partir do começo do século XIV, na Sexta-Feira Santa o rei da Inglaterra, depois de ter depositado moedas sobre um altar, "resgatava-as" colocando em seu lugar uma soma equivalente em não importa que metal sonante; então, mandava qua daquelas primeiras moedas se f{\kern0pt}izessem anéis, os quais em seguida eram dados a certos doentes, especialmente os epiléticos, que f{\kern0pt}icavam curados pelo uso desses anéis chamados cramp-rings.

Marc Bloch faz descrição exemplar do rito e dos gestos que os reis da Inglaterra cumprem (pp. 132-3). Relaciona-os ao uso de diversos talismães, a f{\kern0pt}im de demosntrar "as origens mágicas do rito dos anéis" e evidenciar o fato de que "o nó da ação" residia numa "operação que, de certo modo, é de natureza jurídica: a oferenda das moedas de ouro e de prata e seu resgare por uma soma equivalente" (p. 137). Em seguida, sublinha que esse processo baseado em tradições mágicas é um processo histórico; é "a conquista de uma receita mágica pela realeza miraculosa" (pp. 139 ss.). Sua atenção para os elementos do cerimonial denota uma atitude de antropólogo. Primeiro, dedica-se a determinar os quadros espaciais e temporais,\footnote{A produção dos \textit{cramp-rings} pelos reis da Inglaterra, por exemplo, realiza-se na Sexta-Feira Santa.} essenciais no domínio do sagrado. No dossiê de seus papéis que traz o título "sagração francesa", há na capa duas perguntas: "Onde" e "Quem of{\kern0pt}icia?".

A documentação iconográf{\kern0pt}ica que Marc Bloch reuniu, e cujo estudo ele apenas esboçou, deveria ser completada e metodicamente analisada. A crer nessas imagens, a localização da cerimônia do toque régio dá a impressão de variar de uma igreja a um lugar propriamente real, como o palácio do rei, ou mesmo um lugar ao ar livre -- onde quer que se encontre o rei, o qual cria ao redor de si um tipo de pequeno território régio sagrado. Parece que, por motivos tanto simbólicos quanto práticos, amiúde se recorreu a um meio-termo: capela do palácio real, jardim de uma igreja etc. A ligação com a missa e, às vezes, com a comunhão do rei é frequentemente encontrada. Como observa Marc Bloch, os reis ingleses parecem ter tido mais dif{\kern0pt}iculdade em escapar a uma espécie de absorção num espaço eclesiástico. O tocar as escrófulas torna-se verdadeira liturgia eclesiástica. O milagre dos \textit{cramp-rings} ocorre na capela do palácio na Sexta-Feira Santa, com o altar desempenhando papel central e essencial.

Nesse rito, quem conduz e controla o ritual? Na sagração e na unção do rei da França, é a Igreja, personif{\kern0pt}icada pelo arcebispo de Reims e pelos bispos sufragâneos que o cercam. Nas sessões de toque das escrófulas, o próprio rei não é a um só tempo o agente e o of{\kern0pt}iciante?

Enf{\kern0pt}im, Marc Bloch sublinha a importância que os \textit{objetos sagrados} têm no rito. Em suas anotações, faz o inventário do que chama "elementos" da realeza sagrada: "o sinal régio; a âmbula de s. Tomás Becket, a Santa âmbula de Reims; a pedra de Scone; os leões e os reis; as f{\kern0pt}lores-de-lis (e a aurif{\kern0pt}lama); a comunhão sob as duas espécies; a Santa Lança; a espada; as fórmulas da coroação; o cetro; a coroa; o anel [e os \textit{cramp-rings}]". No momento, Marc Bloch indica muito sagazmente que esses objetos não são utilizados numa sociedade sem história (supondo-se que exista alguma sociedade assim), pois os homens da Idade Média lhes conferem uma identidade histórica adquirida em determinada época, em determinadas circunstâncias. A Santa Âmbula de Reims faz sua entrada na história terrestre no dia do batismo de Clóvis em Reims; a de Thomas Becket, na época do exílio do santo bispo na França; a de Marmoutier, após a queda de s. Martinho no mosteiro, quando um anjo lhe traz um bálsamo divino para curar sua costela quebrada. A Âmbula de Marmoutier foi usada para a sagração de Henrique IV em Chartres em 1594, depois de ter sido trazida a Luís XI em seu leito de morte e (lendariamente) utilizada para a sagração de Luís VI em Orléans em 1108.\footnote{Ver Pierre Gasnault, "La Sainte Ampoule de Marmoutier", em \textit{Analecta Ballandia. Mélanges of{\kern0pt}ferts à Baudouin de Gaif{\kern0pt}f{\kern0pt}ier et François Halkin}, t. 100, 1982, pp. 243-57. Notar-se-á a persistência de um modelo estrutural, com suas variações nos intermediários divinos, sem que a função (no sentido de Propp) mude. No século IX, aparição da Santa Âmbula de Reims (batismo de Clóvis, começo do século VI), trazida por uma \textit{pomba} (Espírito Santo); no começo do século XIV, apresentação da Âmbula de Thomas Becket (ofertada pela \textit{Virgem}, f{\kern0pt}inal do século XII); a Santa Âmbula de Marmoutier, convocada no f{\kern0pt}inal do século XVI, necessita de um dossiê mais rico, porquanto é menos notória (dádiva feita por um \textit{anjo} a s. Martinho, f{\kern0pt}inal do século IV; sagração de Luís VI, começo do século XII; relação com Luís XI, f{\kern0pt}inal do século XV).} O título de uma pasta dos papéis de Marc Bloch é "os objetos da coroação adquirem valor histórico"; e o título de uma f{\kern0pt}icha, "tendência a ver no objeto simbólico do início um objeto histórico". O historiador Marc Bloch colhia a profunda tendência do cristianismo a inscrever-se na historicidade. Tradição, folclore, símbolos são transportados à história.

Daí ele passa com naturalidade ao problema mais geral da "realeza miraculosa e sagrada" no Ocidente medieval. Então, abre primeiro o dossiê da realeza sacerdotal. O resultado é exíguo. Pois, se em Bizâncio o basileu conseguiu dominar o espiritual e o secular, não houve no Ocidente nada de semelhante, nada de cesaro-papismo. Os reis (e o imperador) oscilaram entre duas atitudes ou tentaram combiná-las. Ou distinguirem claramente o espiritual e o secular e tornarem-se os independentes senhores do secular, o que denominarei averroísmo político (segundo a doutrina das duas verdades, a da fé e a da razão). Ou -- assim como os papas que, por causa do poder de absolver ou não em razão do pecado (\textit{ex ratione peccati}), reinvindicaram um direito de vigiar o secular -- adquirem um poder no domínio espiritual, obtendo certo estatuto \textit{sacerdotal}. Aqui, Marc Bloch chama a atenção para o fato de que, mais que a noção de \textit{rex-sacerdos} avançada pelos teólogos e teóricos da Querela do Sacerdócio e do Império, é no domínio litúrgico (mediante a análise dos tratados ou, melhor ainda, dos rituais) que se podem perceber as analogias pelas quais os reis tentaram inf{\kern0pt}iltra-se na hierarquia eclesiástica. Do lado da Igreja, a tendência era conf{\kern0pt}inar os monarcas num papel próximo ao de \textit{subdiáconos}, ao passo que o cerimonial sagrado revela certo esforço do rei e de sua corte para modelar sua "ordenação" pela do bispo. Mas é um estudo em que quase tudo está por fazer.

~\\ \large \textit{LEGENDAS} ~\\

Em seguida, Marc Bloch percorre o caminho das \textit{legendas} que ilustraram a monarquia sagrada medieval e, mais particularmente, o "ciclo monárquico francês". Aqui, Marc Bloch reúne um feixe de crenças, ligadas pelo caráter sobrenatural de determinado número de insígnias régias que deram origem a legendas, e acrescenta a elas o toque das escrófulas: "A Santa Âmbula, as f{\kern0pt}lores-de-lis trazidas do céu, a aurif{\kern0pt}lama, também de origem celeste; acrescentemos o dom de curar; teremos então o feixe maravilhoso que os apologistas da realeza capetíngia deviam daí em diante oferecer, sem tréguas, à admiração da Europa" (p. 175). Assim, ao lado das insígnias régias propriamente ditas, das \textit{regalia} que, ao contrário da Santa Âmbula conservada na abadia Saint-Rémi de Reims, são guardadas na abadia real de Saint-Denis (a coroa; o gládio; as esporas de ouro; o cetro dourado; a virgem com mão de marf{\kern0pt}im; os calçoes de seda violeta enfeitados com lírios de ouro; a dalmática "com que os subdiáconos vestem-se para a missa"; a sobrecota sem capuz, igualmente violeta),\footnote{Onde estão conservadas pelo menos desde meados do século XII, pois (contrariamente ao que pretendeu P. E. Schramm f{\kern0pt}ixando em 1260 o momento em que o rei conf{\kern0pt}iou-as a s. Dnis) elas decerto estavam lá quando da sagração de Filipe Augusto, em 1179. Cf. \textit{Recueil des historiens de la France}, t. 12. p. 215, e E. Berger, "Annales de Saint-Denis", em \textit{Bibliothèque de l'École des Chartres}, t. 40, 1879, pp, 279-88. Sobre a aurif{\kern0pt}lama, ver Ph. Contamine, "L'orif{\kern0pt}lamme de Saint-Denis aux  XIV\textsuperscript e\textnormal et XV\textsuperscript e \textnormal siècles. Étude de symbolique religieuse et royale", em \textit{Annales de l'Est}, 1973, n. 3, pp. 179-244.} há os objetos sobrenaturais vindos do céu e o poder de curar. Esses objetos e esse poder colocam o rei em comunicação direta com Deus, mas o intermediário eclesiástico é (até certo ponto) mantido: a Santa Âmbula foi trazida a s. Remígio; o abade de Saint-Rémi é quem a conserva, traz e leva de volta no dia da sagração; o arcebispo de Reims é quem unge o rei. Ademais, se o arcebispo Hinemar de Reims, o primeiro a relatar a legenda por escrito (no século IX), tomou-a emprestada, como pensa Marc Bloch, às tradições folclóricas locais, então o arcebispo decerto registrou o milagre para servir, antes de tudo, às pretensões de supremacia eclesiástica da Igreja de Reims e para af{\kern0pt}irmar, à moda carolíngia, o controle da monarquia pela Igreja.

Marc Bloch não compara o poder de curar dos reis da França e da Inglaterra na Idade Média ao dos chefes carismáticos de outras sociedades, pois já percebe os limites de um método comparativista ao qual, no entanto, é levado. Servindo-se de seu principal gua em antropologia, Frazer, ele evoca as crenças e práticas das tribos da Oceania, e os poderes dos chefes das ilhas Tonga, na Polinésia. Mas trata-se de um caso isolado, e Marc Bloch formula uma das leis mestras do bom comparativismo: "O estudo das tribos da Oceania esclarece a noção de realeza sagrada, tal como ela f{\kern0pt}loresceu sob outros céus, na Europa antiga ou mesmo medieval; mas não poderíamos esperar encontrar na Europa todas as instituições da Oceania. [...] Entre os primeiros missionários, muitos acreditavam reencontrar nos 'selvagens', mais ou menos apagadas, todas as espécies de concepções cristãs. Evitemos cometer o erro inverso e não transportemos para Paris ou para Londres os antípodas por inteiro" (pp. 70).

Depois, Marc Bloch faz um desvio até duas legendas que f{\kern0pt}icaram à margem da cristianização do legendário monárquico: o sinal régio e a atitude dos leões para com os reis. Segundo a crença estritamente popular, não admitida pela Igreja, o rei da França, à semelhança de outros soberanos, é dotado de um sinal na pele, uma mancha, um nevo em forma de cruz, de cor vermelho-vivo, quase sempre sobre o ombro direito, mas raramente sobre o peito. Muito provavelmente, foi o sinal que Carlos VII mostrou em privado a Joana d'Arc em Chinon, para provar-lhe que era de fato o f{\kern0pt}ilho legítimo de Carlos VI e não um bastardo. Essa crença é encontrada tanto na Antiguidade helênica quanto nas pretensões de certos charlatães da Europa moderna. Por outro lado, o bom povo acredita que "os leões jamais ferem um verdadeiro rei". E em 1340 um dominicano, embaixador de Eduardo III em Veneza, conta ao doge que o rei da Inglaterra "teria aceitado reconhecer Filipe de Valois como rei da França se esse príncipe, havendo-se exposto a leões famintos, saísse ileso das garras destes" (pp. 40 e 186-7).

Enf{\kern0pt}im, ao término de longo estudo pessoal e original, Marc Bloch analisa a contaminação (fundamental fenômeno do folclore, que o historiador deve acolher em seu próprio domínio) entre o culto de um santo e o rito régio da cura das escrófulas. Desde o começo do século X, existia em Corbeny, no Aisne, a devoção popular a um santo vindo do Cotentin: Marculf ou Marcoul, o qual, aparentemente no século XIII, especializou-se também na cura dos escrofulosos, provavelmente em virtude de um jogo de palavras etimológico com \textit{mar}, mal, e \textit{cou}(\textit{l}), pescoço. Esse poder foi relacionado ao poder dos reis, e dois cultos uníram-se. do século XIV ao século XVII, após a sagração os reis da França, exceto Henrique IV, f{\kern0pt}izeram um desvio por Corbeny para receber nas mãos a cabeça (o crânio) do santo e, em seguida, tocar escrofulosos com um poder aumentado pelo do santo. Luís XIV e seus sucessores mandaram vir a Reims no momento da sagração o relicário que continha os restos do santo.

À contaminação entre o culto de s. Marcoul e o milagre régio Marc Bloch acrescenta uma terceira crença popular, que esteve historicamente ligada às duas primeiras. Em alguns lugares, acreditava-se que o sétimo de uma série contínua de f{\kern0pt}ilhos varões possuía poderes de mágico e, especialmente, de curandeiro. Por assimilação ao poder de curar dos reis, dizia-se que esses sétimos f{\kern0pt}ilhos não apenas tinham o dom de sarar os escrofulosos mas também nasciam com uma marca distintiva no corpo. Por f{\kern0pt}im, torna-se costume, que, antes de exercer seus dons, esses sétimos f{\kern0pt}ilhos façam uma peregrinação a Corbeny, até as relíquias de s. Marcoul. Marc Bloch, o qual reuniu um volumoso dossiê sobre essa crença nas províncias francesas, na Europa e até na América (cherokees), interessou-se sobretudo pelo mecanismo de interpretação dos três fenômenos a por esse encontro histórico entre uma crença popular e as práticas toleradas ou integradas pelos clérigos.

~\\ \large \textit{O FIM DO MILAGRE} ~\\

Marc Bloch termina a mais longa parte de seu livro -- em que cruzou cronologia e temática, ainda um bom método de historiador -- com o estudo das vicissitudes do milagre régio entre o século XVI e o século XVIII, "no tempo das lutas religiosas e do absolutismo" e, depois, na época do "declínio" e da "morte" dessa crença.

É demonstração de que em novos contextos históricos uma estrutura, o toque régio, muda de lugar e de signif{\kern0pt}icado sem mudar essencialmente de forma. A morte do rito: na Inglaterra, ele sobre fortemente o ataque do protestantismo e desaparece com a mudança dinástica de 1714; na França, seu f{\kern0pt}im coincide com a Revolução e a queda da monarquia, não obstante o breve e anacrônico ressuscitamento que teve em 1825, na sagração de Carlos X. Ora, o fundamental não está nesses acontecimentos, por mais importantes que sejam. Um fenômeno histórico, sobretudo uma crença, um fato mental, raramente é assassinado. Morre mais ou menos lentamente, seguindo o ritmo da mudança, tanto da mentalidade quanto das condições em que essa mentalidade apareceu.

Aqui, Marc Bloch abandona os ritos, os gestos, as imagens; não recorre mais ao folclore, à etnograf{\kern0pt}ia, à medicina. As "coisas profundas", a "psicologia coletiva" sofrem a inf{\kern0pt}luência decisiva da evolução intelectual das elites. O que matou o milagre régio foi o espírito "racionalista"{ }que, a partir do século XVII, procurou encontrar uma explicação racional para o portento -- até que as Luzes, no século XVIII, renunciam a essa pesquisa e proclamam que pura e simplesmente o milagre não existe. Não podendo ser elucidado mediante alguma razão natural (o sangue, por exemplo), o milagre régio desaparece da crença erudita, junto com todos os outros milagres, com "toda uma concepção de universo"{ }à qual era "aparentado". Sempre lúcido, Marc Bloch vê que na "opinião comum"{ }setecentista há uma cisão entre os espíritos esclarecidos e o "vulgo", o qual continua a acreditar numa "ação miraculosa" (p. 270).

~\\ \large \textit{EXPLICAÇÃO: UM 'ERRO COLETIVO'} ~\\







\end{document}