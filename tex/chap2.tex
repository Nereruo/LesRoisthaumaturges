% Livro 1 - Capítulo 2 - As origens do poder curativo dos reis: a realeza sagrada nos primeiros séculos da Idade Média
%Transcrição: Pedro Bernardinelli
% @bernardinelli
% pedro@bernardinelli.org


\chapter{As origens do poder curativo dos reis: a realeza sagrada nos primeiros séculos da Idade Média} 


\section{A evolução da realeza sagrada: a sagração}
O problema que agora exige nossa atenção é duplo. O milagre régio apresenta-se sobretudo como a expressão de certo conceito de poder político supremo. Desse ponto de vista, explicá-lo será correlacioná-lo ao conjunto de idéias e de crenças de que o milagre régio foi uma das manifestações mais características --- pois não é exatamente o princípio de toda a ``explicação'' científica fazer um caso particular encaixar-se num fenômeno mais geral? Mas, tendo conduzido nossa pesquisa até tal ponto, não teremos ainda terminado nosso trabalho. Parando aí, deixaríamos escapar justamente o particular; faltará entender as razões pelas quais o rito curativo, derivado de um movimento de pensamentos e de sentimentos comuns a toda uma parte da Europa, surgiu em determinado momento e não em outro, na França e na Inglaterra e não em outro lugar. Em suma, temos, de um lado, as causas profundas e, de outro, a ocasião, o empurrãozinho que chama para a vida uma instituiçõ que, desde longa data, estava latente nos espíritos.

Mas talvez alguém pergunte: é verdadeiramente necessária uma longa investigação para descobrir as representações coletivas que estão na origem do toque das escrófulas? Não é óbvio que esse rito, aparentemente tão singular, não foi nas sociedades medievais e modernas senão o último eco dessas crenças ``primitvas'' que hoje, graças ao estudo dos povos selvagens, a ciência conseguiu reconstruir? Para compreender o rito do toque, não é suficiente percorrer os grandes compêndios levantados com tanto cuidado e talento por sir James Frazer, folhear \emph{O maro de ouro} ou \emph{As origens mágicas da realeza}? ``Que teria dito Luís \sc{XIV}'', escreve o sr. Salomon Reinach, ``se alguém lhe provasse que, tocando as escrófulas, ele seguia o exemplo de um chefe polinésio?''\footnote{\emph{Cultes, mythes et religions}, \sc{II}, p. 21.} E Montesquiei (sob a máscara do persa Usbeck) já falava do mesmo príncipe: ``Esse rei é um grande mágico; exerce seu domínio sobre o próprio espírito de seus súditos [\ldots] Chega até a fazê-los acreditar que os cura de todas as espécies de males tocando-os, tão grande é a força e o poder que tem sobre os espíritos''.\footnote{\emph{Lettres persanes}, 1. 24.} No pensamento de Montesquieu, a palavra mágico era apenas uma expressão irônica. Hoje, de bom grado damos a ela seu sentido pleno. Usei como epígrafe essa pequena frase do Montesquieu; com mais justiça ainda, ela poderia ter sido inscrita no frontispício das belas obras de sir James Frazer, que nos ensinaram a perceber entre certas concepções antigas sobre a natureza das coisas e as primeiras instituições políticas da humanidade vínculos por longo tempo ignorados. Sim, o milagre das escrófulas tem, incontestavelmente, parentesco com todo um sistema psicológico que, por uma dupla razão, se pode qualificar de ``primitivo'': porque traz a marca de um pensamento ainda pouco evoluído e de todo mergulhado no irracional; e porque o encontramos em estado especialmente puro nas sociedades que convencionamos chamar de ``primitivas''. Mas, após havermos dito isso, que teremos feito senão indicar aproximadamente o gênero das representações mentais para as quais é conveniente dirigir nossa pesquisa? A realidade histórica é menos simples e mais rica que semelhantes fórmulas.
\section{O poder curativo do sagrado}

\section{A política dinástica dos primeiros capetíngios e de Henrique I Beauclerc}